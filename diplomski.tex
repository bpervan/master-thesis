\documentclass[times, utf8, diplomski]{fer}
\usepackage{booktabs}

\begin{document}

% TODO: Navedite broj rada.
\thesisnumber{1136}

% TODO: Navedite naslov rada.
\title{Preporučiteljski sustavi u sveprisutnom računarstvu}

% TODO: Navedite vaše ime i prezime.
\author{Branimir Pervan}

\maketitle

% Ispis stranice s napomenom o umetanju izvornika rada. Uklonite naredbu \izvornik ako želite izbaciti tu stranicu.
\izvornik

% Dodavanje zahvale ili prazne stranice. Ako ne želite dodati zahvalu, naredbu ostavite radi prazne stranice.
\zahvala{Ovdje dolazi zahvala}

\tableofcontents

\chapter{Uvod}


\chapter{Preporučiteljski sustavi}
\section{Uvod u preporučiteljske sustave}
Preporučiteljski sustavi su, ukratko rečeno, skup programskih alata i tehnika
koji krajnjem korisniku pružaju preporuku za neki predmet koji mu je na neki
način u interesu [2]. Za rad svakog preporučitelja bitna su dva skupa: skup
predmeta preporuke i skup korisnika sustava.

Predmet se shvaća generički i on može varirati ovisno o kontekstu primjene,
primjerice, artikli u internet trgovini, knjige u digitalnim knjižnicama, pjesme
i filmovi na multimedijalnim servisima, rezultati pretraživanja na tražilicama,
osobe na društvenim mrežama i servisima za upoznavanje i sl. Nadalje, predmeti
se mogu opisati pripadajućim atributima uključivo s kompleksnosti i vrijednosti,
dok vrijednost može biti pozitivna ako postoji potencijal korisnosti za
korisnika, ili negativna ako predmet ne odgovara korisniku. Ti atributi opet mogu 
varirati u ovisnosti o samom predmetu preporuke, primjerice žanr ukoliko je predmet 
glazba, ili boja ukoliko je predmet takav da ga mu se taj atribut može pridijeliti.

S druge strane, korisnici sustava imaju različite scenarije korištenja
preporučitelja od kojih su osnovni filtriranje neželjenog sadržaja iz velikih
baza podataka i savjetovanje pri nedostatku vlastite kompetencije za izbor sadržaja. 
Interakcija korisnika sa sustavom omogućuje praćenje njegovih odabira, te kroz analizu 
profila korisnika i njegovih osobnih preferencija stvaranje modela za preporuku predmeta 
na nekoliko načina. Podaci koje korisnik ostavlja u sustavu u osnovi se mogu podijeliti 
u dva skupa: implicitni i eksplicitni [3]. Implicitni podaci su oni podaci za koje korisnik nije 
izravno svjestan da ih ostavlja. 
Neki od tih podataka su npr. demografski podaci, točnije, šire područje iz kojeg korisnik koristi sustav a 
jednostavno se doznaje iz baze podataka dodijeljenih područja (eng. scope) IP adresa. 
Pod implicitne podatke spadaju i akcije korisnika tokom rada sa sustavom koje nemaju 
izravne veze sa ocjenjivanjem.
Eksplicitni su pak oni podaci koje korisnik ostavlja s namjernom, primjerice koristeći ankete o svojim 
preferencijama ili odgovarajući na upite o pojedinim predmetima. Na ovaj način provodi se svojevrsno 
treniranje preporučitelja.

Preporučiteljski sustavi razlikuju se prema načinu filtriranja i analiziranja informacija, 
a razlikujemo četiri osnovna načina [3]:
\begin{enumerate}
  \item Preporučivanje neovisno o korisniku (eng. \textit{Non -- personalized
  recommenders})
  \item Preporučivanje zasnovano na sadržaju (eng. \textit{Content -- based
  recommendation})
  \item Preporučivanje zasnovano na suradnji (eng. \textit{Collaborative
  recommendation})
  \item Hibridne tehnike preporučivanja
\end{enumerate}

Razvoj preporučiteljskih sustava započeo je devedesetih godina prošlog stoljeća,
a nemalo je populariziran 2006. g. svojevrsnim natjecanjem „The Netflix Prize” 
kada je poznati pružatelj multimedije na zahtjev ponudio nagradu od
\$$1,000,000$ američkih dolara za tim koji razvije preporučitelj bolji od tada
postojećeg \glqq Cinematch \grqq za određeni postotak [5].
Ovo je ostavilo velik utjecaj na razvoj preporučitelja prvenstveno zbog činjenice da je u uvjetima 
natjecanja navedeno da rezultati i principi rada razvijenih preporučitelja 
moraju biti javno objavljeni i dostupni.

\section{Nepersonalizirani preporučitelji}
\section{Personalizirani preporučitelji}
\subsection{Preporučitelji zasnovani na sadržaju}
\subsection{Preporučitelji zasnovani na suradnji}
\subsubsection{Korisnik -- Korisnik}
\subsubsection{Predmet -- Predmet}

\subsection{Hibridni preporučitelji}
\subsection{Moguća područja primjene}

\chapter{Modeliranje podataka}
\section{Modeliranje korisnika}
\section{Modelrianje predmeta}

\chapter{Problem vremena i prostora}

\chapter{Razvoj algoritma i radnog okvira}

\chapter{Testiranje i evaluacija}
\section{Metodologija}
\section{Testiranje}
\section{Evaluacija preporučitelja}

\chapter{Zaključak}
Zaključak.

\bibliography{literatura}
\bibliographystyle{fer}

\begin{sazetak}
Sažetak na hrvatskom jeziku.

\kljucnerijeci{Ključne riječi, odvojene zarezima.}
\end{sazetak}

% TODO: Navedite naslov na engleskom jeziku.
\engtitle{Recommender systems in ubiquituous computing}
\begin{abstract}
Abstract.

\keywords{Keywords.}
\end{abstract}

\end{document}
